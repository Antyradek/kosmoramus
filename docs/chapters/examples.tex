\chapter{Przykłady}
Przykładowe moce, chipy, urządzenia, itp.
\kosmoramus{} jest grą o kreatywności, Mistrz Gry może specjalnie dla waszej sesji dodać sobie coś nowego.

Także są tutaj przykładowi gracze z których możecie sobie wybrać.

\section{Moce}
Zmienna $p$ jest obecnym poziomem cechy.

\subsection{Łańcuch}
Postać może wytworzyć i modyfikować łańcuch.
Łańcuch ma określoną długość i wytrzymałość.
W kilku turach można tworzyć bardzo długi łańcuch poprzez wytworzenie i połączenie fragmentów.
Można go telepatycznie zniknąć.
\begin{description}
	\item[SM] Wytworzenie łańcucha o długości $l$ i wystrzelenie w powietrze z prędkością ruchu ręki (test \aba{}).
	\item[DM] Ruszanie łańcuchem telepatycznie w odległości $d$, przerwanie go, modyfikowanie.
	\item[RM] Wykrycie położenia i dotyku łańcucha.
\end{description}
\begin{itemize}
	\item Długość wytworzonego łańcucha $l = 10p \ \si{\centi\metre}$
	\item Wytrzymałość $F = 20p \ \si{\newton}$
	\item Odległość $d = p \ \si{\metre}$
\end{itemize}

\subsection{Tarcza}
Wytworzenie okrągłej tarczy o średnicy \SI{1}{\metre}.
Można nią ruszać w granicach maksymalnej siły, przekroczenie natychmiast niszczy tarczę.
Można telepatycznie zniknąć ją przed czasem.
\begin{description}
	\item[SM] Wytworzenie tarczy w odległości $d$ od siebie.
	\item[DM] Ruch tarczą z siłą nie większą, niż $F$.
	\item[RM] Wykrycie pozycji i dotyku na tarczy.
\end{description}
\begin{itemize}
	\item Maksymalna siła $F = 10p \ \si{\newton}$
	\item Czas zaniku $t = p \ \text{tur}$
	\item Odległość $d = p \ \si{\metre}$
\end{itemize}

\subsection{Modyfikacja grawitacji}
Gracz może nadać jakąś siłę w dowolnym kierunku dla pojedynczych obiektów.
Pozwala to na rzucenie czegoś o ścianę, zmniejszenie ciężaru, a nawet latanie.
Da się poruszać lekkimi przedmiotami w odległości \SI{1}{\metre}.
Można wykryć prędkość poruszania się obiektu i jego pozycję.
\begin{description}
	\item[SM] Nadanie siły $F$ obiektowi przez określony czas $t$.
	\item[DM] Telekineza w małej odległości.
	\item[RM] Wykrycie prędkości, obrotu i pozycji obiektu.
\end{description}
\begin{itemize}
	\item Siła $F = 10p \ \si{\newton}$
	\item Kąt w stosunku do grawitacji $\alpha = 0,1\pi p \ \si{\radian}$
	\item Czas działania $t = p \ \text{tur}$
\end{itemize}

\subsection{Frisbee}
Postać wytwarza i rzuca okrągłe i płaskie pierścienie (jak frisbee) z ostrymi krawędziami.
Średnica zewnętrzna nie może przekroczyć \SI{1}{\metre}, a wewnętrzna jest dowolna.
Krawędzie nie tną stwórcy, ale zadają obrażenia wszystkim innym w wielkości $k$ \abzyc{}.
Celność jest osiągana testem \aba{}.
Na pierścienie nie działa grawitacja ale nie są w stanie utrzymać więcej, jak \SI{1000}{\newton} (jednego człowieka).
Nie można ich obracać w innej płaszczyźnie, niż są obecnie.
Nie można ich zniknąć, znikają same po 10 rundach.
\begin{description}
	\item[SM] Wytworzenie pierścienia i rzucenie nim z prędkością $v$.
	\item[DM] Ruch pierścieniem w płaszczyźnie wytworzenia z maksymalną prędkością $v$, w odległości $d$.
	\item[RM] Odczucie, co pierścień dotyka.
\end{description}
\begin{itemize}
	\item Odległość sterowania $d = p \ \si{\metre}$
	\item Prędkość wyrzutu $v = p \ \si{\metre\per\second}$
	\item Ostrość $k = p \ \text{\abzyc{}}$
\end{itemize}

\subsection{Kolec}
Gracz strzela kolcem.
Dodajemy $c$ do kostki o tyle ułatwiając sobie test.
Postać czuje, gdy ktoś dotyka kolca.
Może go telepatycznie zniknąć.
\begin{description}
	\item[SM] Wystrzelenie kolca odbierającego $k$ \abzyc{}, z prędkością $v$.
	\item[DM] Zniknięcie, nadanie siły testem \abs{}.
	\item[RM] Odczucie pozycji kolca i czego dotyka.
\end{description}
\begin{itemize}
	\item Prędkość strzału $v = p \ \si{\metre\per\second}$
	\item Naprowadzanie $c = p$
	\item Ostrość $k = p \ \text{\abzyc{}}$
\end{itemize}

\subsection{Uderzenie}
Gracz nadaje impuls tak silny, jak kopnięcie nogą.
Może nim rzucić, zburzyć coś w odpowiednim kształcie i zbadać strukturę.
\begin{description}
	\item[SM] Impuls na stożku o kącie $\alpha$ i zasięgu $l$, w odległości $d$ od postaci.
	\item[DM] Wytworzenie określonego kształtu impulsu przy dotyku ręką.
	\item[RM] Dotknięcie i zbadanie struktury materiału.
\end{description}
\begin{itemize}
	\item Kąt $\alpha = 0.2p\pi \ \si{\radian}$
	\item Zasięg $l = 0.5p \ \si{\meter}$
	\item Odległość $d = p \ \si{\meter}$
\end{itemize}

\subsection{Echo}
Postać generuje fale soniczne, dzięki którym widzi w ciemności, a także potrafi wprawić w wibracje jakiś obiekt.
\begin{description}
	\item[SM] Wprawienie w wibracje obiektu w zasięgu $d$, którego pozycja jest znana. Wykonaj test \aba{} i dodaj $c$ do kostki.
	\item[DM] Dokładne poznanie kształtu obiektów w kuli o promieniu $r$, w zasięgu $d$.
	\item[RM] Poznanie przybliżonego kształtu otoczenia w zasięgu $d$.
\end{description}
\begin{itemize}
	\item Zasięg $d = p \ \si{\meter}$
	\item Promień kuli $r = p \ \si{\centi\meter}$
	\item Celność $c = p$
\end{itemize}

% elektrostatyka
% leczenie

\section{Przedmioty}
\begin{description}
	\item[Wkłady antymaterii] Podstawowe zasilanie broni i urządzeń.
	\item[Wkłady do szyfratora] Jedyne i niepowtarzalne do jedynej i niepowtarzalnej broni.
\end{description}

\section{Bronie}
Pif-paf, ziuu, (dźwięk zamarzającej czasoprzestrzeni).
\subsection{Mostownik \abs{}}

\begin{tabular}{lr}
Wkłady & Antymateria \\
Magazynek & 5 \\
Załadowanie na turę & 1 \\
\end{tabular}

Wielka deska długości metra.
Tworzy rzucane, paraboliczne mosty do celu.
Most utrzymuje się w powietrzu zużywając jeden nabój na rundę.
\begin{enumerate}
	\item Most się tworzy, ale wije się w górę i w dół.
	\item Most się tworzy, ale wije się na boki.
	\item Most się tworzy, ale skręca się wzdłuż, jak wstążka.
	\item Most się tworzy, ale w połowie się urywa.
	\item Nic się nie dzieje.
	\item Tworzą się jedynie fragmenty.
	\item Tworzą się tylko granice, jak szyny. 
	\item Most jest dziwnie lepki.
	\item Most jest tak śliski, że nie da się go używać.
	\item Most tworzy się w dziwnym kierunku \dii{}, czy przetnie strzelającego na pół.
\end{enumerate}

\subsection{Szyfrator \abt{}}

\begin{tabular}{lr}
Wkłady & Szyfrator \\
Magazynek & 1 \\
Załadowanie na turę & 0,5 \\
\end{tabular}

Najwspanialsze urządzenie inżynierii potworów.
Podręczy pistolet, który zamraża wroga w czasie.
Można go odmrozić tym samym wkładem.
Działa to na zasadzie szyfrowania czasoprzestrzeni, klucz do odszyfrowania składowany jest w naboju.

Zmiana naboju trwa dwie rundy, najpierw trzeba wyciągnąć stary i włożyć nowy.
Wyrzucenie starego naboju wiąże się ze śmiercią zamrożonego (zostaje taki na zawsze) i dodaje 10 \abgrz{}.
Strzelenie ponownie tym samym nabojem nadpisuje klucz i dodaje 10 \abgrz{}.
Jeśli nie trafiło się w poprzedniej rundzie, można strzelać ponownie bez przeładowania.

Przy niezdaniu testu cechy nie ma żadnych efektów ubocznych!

\section{Chipy aktywne}
Rurka do oddychania pod wodą wkręca mi się w głowę.

\subsection{Ostrza z ręki \abh{}, 3 \abnkp{}}
Trochę, jak Volverine, ale lepiej.
Gracz wysuwa trzy krótkie kolce z przedramienia.
\begin{enumerate}
	\item Tym razem wysuwa ci się tylko środkowy.
	\item Tym razem wysuwają ci się tylko boczne.
	\item Środkowy kolec wypada na podłogę.
	\item Środkowy kolec wystrzeliwuje, \dii{} czy kogoś trafi.
	\item Kolce przestają się wysuwać w ogóle.
	\item Kolce przestają się chować w ogóle.
	\item Wszystkie trzy kolce wystrzeliwują we wszystkie strony.
	\item Kolce wysuwają się w drugą stronę, w rękę, -5 \abzyc{}.
	\item Zaczynają ciągle szybko wsuwać się i wysuwać.
	\item Wysuwają się od teraz tylko do połowy.
\end{enumerate}




% typy graczy z zestawami cech, chipami, mocą itp 
% Przykładowe chipy wraz z opisem uszkodzeń, efektów, cech

% Przykładowe mutacje

% Przedmioty

% Urządzenia

% Wiedza