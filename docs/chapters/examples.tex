\chapter{Przykłady}
Przykładowe moce, chipy, urządzenia, itp.
\kosmoramus{} jest grą o kreatywności, Mistrz Gry może specjalnie dla waszej sesji dodać sobie coś nowego.

Także są tutaj przykładowi gracze z których możecie sobie wybrać.

\section{Moce}
Zmienna $p$ jest obecnym poziomem cechy.
Każdą wytworzoną materię można telepatycznie zniknąć w jednej turze.

\subsection{Łańcuch}
Postać może wytworzyć i modyfikować łańcuch.
Łańcuch ma określoną długość i wytrzymałość.
W kilku turach można tworzyć bardzo długi łańcuch poprzez wytworzenie i połączenie fragmentów.
\begin{description}
	\item[SM] Wytworzenie łańcucha o długości $l$ i wystrzelenie w powietrze z prędkością ruchu ręki (test \aba{}).
	\item[DM] Ruszanie łańcuchem telepatycznie w odległości $d$, przerwanie go, modyfikowanie, zniknięcie.
	\item[RM] Wykrycie położenia, stanu i dotyku łańcucha.
\end{description}
\begin{itemize}
	\item Długość wytworzonego łańcucha $l = 10p \ \si{\centi\metre}$
	\item Wytrzymałość $F = 20p \ \si{\newton}$
	\item Odległość modyfikacji i wyczucia $d = p \ \si{\metre}$
\end{itemize}

\subsection{Tarcza}
Wytworzenie okrągłej tarczy o średnicy \SI{1}{\metre}.
Można nadawać jej siłę w określonych limitach, przekroczenie natychmiast niszczy tarczę.
Na tarczę nie działa grawitacja.
\begin{description}
	\item[SM] Wytworzenie tarczy w odległości $d$ od siebie.
	\item[DM] Ruch tarczą z siłą nie większą, niż $F$, przedwczesne zniknięcie.
	\item[RM] Wykrycie pozycji i dotyku na tarczy.
\end{description}
\begin{itemize}
	\item Maksymalna siła $F = 10p \ \si{\newton}$
	\item Czas zaniku $t = p \ \text{tur}$
	\item Odległość $d = p \ \si{\metre}$
\end{itemize}

\subsection{Modyfikacja grawitacji}
Gracz może nadać jakąś siłę w dowolnym kierunku dla pojedynczych obiektów.
Pozwala to na rzucenie czegoś o ścianę, zmniejszenie ciężaru, a nawet latanie.
Da się poruszać lekkimi przedmiotami w odległości \SI{1}{\metre}.
Można wykryć prędkość poruszania się obiektu i jego pozycję.
\begin{description}
	\item[SM] Nadanie siły $F$ obiektowi przez określony czas $t$.
	\item[DM] Telekineza małych obiektów w małej odległości.
	\item[RM] Wykrycie prędkości, obrotu i pozycji obiektu.
\end{description}
\begin{itemize}
	\item Siła $F = 10p \ \si{\newton}$
	\item Kąt w stosunku do grawitacji $\alpha = 0,1\pi p \ \si{\radian}$
	\item Czas działania $t = p \ \text{tur}$
\end{itemize}

\subsection{Pierścienie}
Postać wytwarza i rzuca okrągłe i płaskie pierścienie (jak frisbee) z ostrymi i twardymi krawędziami.
Średnica zewnętrzna nie może przekroczyć \SI{1}{\metre}, a wewnętrzna jest dowolna.
Krawędzie nie tną stwórcy, ale zadają obrażenia wszystkim innym w wielkości $kv$ \abzyc{}.
Na pierścienie nie działa grawitacja ale nie są w stanie utrzymać więcej, jak \SI{1000}{\newton} (jednego człowieka).
Nie można ich obracać w innej płaszczyźnie, niż są ustawione obecnie.
Można się zastawić pierścieniem, jak tarczą, ale wytrzymuje on tylko jedno uderzenie.
\begin{description}
	\item[SM] Wytworzenie pierścienia i rzucenie nim z prędkością $v$.
	\item[DM] Ruch pierścieniem w płaszczyźnie wytworzenia z maksymalną prędkością $v$, w odległości $d$, zniknięcie.
	\item[RM] Odczucie pozycji i czego pierścień dotyka.
\end{description}
\begin{itemize}
	\item Odległość sterowania $d = p \ \si{\metre}$
	\item Prędkość wyrzutu $v = p \ \si{\metre\per\second}$
	\item Ostrość $k = p \ \text{\abzyc{}}$
\end{itemize}

\subsection{Kolec}
Gracz strzela kolcem.
Po zdaniu testu mocy, dodajemy $c$ do kostki o tyle ułatwiając sobie test cechy.
Postać czuje, gdy ktoś dotyka kolca.
\begin{description}
	\item[SM] Wystrzelenie kolca odbierającego $kv$ \abzyc{}.
	\item[DM] Zniknięcie, nadanie siły testem \abs{}.
	\item[RM] Odczucie pozycji kolca i czego dotyka.
\end{description}
\begin{itemize}
	\item Prędkość strzału $v = p \ \si{\metre\per\second}$
	\item Naprowadzanie $c = p$
	\item Ostrość $k = p \ \text{\abzyc{}}$
\end{itemize}

\subsection{Uderzenie}
Gracz nadaje impuls tak silny, jak kopnięcie nogą.
Może nim rzucić, zburzyć coś w odpowiednim kształcie i zbadać strukturę.
\begin{description}
	\item[SM] Impuls na stożku o kącie $\alpha$ i zasięgu $l$, w odległości $d$ od postaci.
	\item[DM] Wytworzenie określonego kształtu impulsu przy dotyku ręką.
	\item[RM] Dotknięcie i zbadanie struktury materiału na małą głębokość.
\end{description}
\begin{itemize}
	\item Kąt $\alpha = 0,2p\pi \ \si{\radian}$
	\item Zasięg $l = 0,5p \ \si{\meter}$
	\item Odległość $d = p \ \si{\meter}$
\end{itemize}

\subsection{Echo}
Postać generuje fale soniczne, dzięki którym widzi w ciemności, a także potrafi wprawić w wibracje jakiś obiekt z zamierzonym skutkiem.
\begin{description}
	\item[SM] Wprawienie w wibracje obiektu w zasięgu $d$, którego pozycja jest znana. Dodaj $c$ do kostki przy teście cechy.
	\item[DM] Dokładne poznanie kształtu obiektów w kuli o promieniu $r$, w zasięgu $d$.
	\item[RM] Poznanie przybliżonego kształtu otoczenia w zasięgu $d$ od ciebie.
\end{description}
\begin{itemize}
	\item Zasięg $d = p \ \si{\meter}$
	\item Promień kuli $r = p \ \si{\centi\meter}$
	\item Celność $c = p$
\end{itemize}

% TODO leczenie

% TODO łączenie życia

\section{Przedmioty}
\begin{description}
	\item[Wkłady antymaterii] Podstawowe zasilanie broni i urządzeń.
	\item[Wkłady do szyfratora] Jedyne i niepowtarzalne do jedynej i niepowtarzalnej broni.
	\item[Wkłady do baniecznicy] Mogą mieć dowolne moce, także ujemne,
\end{description}

\section{Bronie}
Pif-paf, ziuu, (dźwięk zamarzającej czasoprzestrzeni).
W swoim obidzie użytkownik może całkowicie opróżnić magazynek (na przykład aby zmienić kolejność wkładów), ale dopiero w następnym może załadować jeden nowy wkład.

\subsection{Mostownik \abs{}}

\begin{tabular}{rl}
Wkłady & Antymateria \\
Magazynek & 5 \\
Załadowanie na turę & 1 \\
Obrażenia & 10 \\
\end{tabular}

Wielka deska długości metra.
Tworzy rzucane, paraboliczne mosty do celu.
Most utrzymuje się w powietrzu zużywając jeden nabój na rundę.
Jeśli kogoś trafi, zadaje mu obrażenia.
\begin{enumerate}
	\item Most się tworzy, ale wije się w górę i w dół.
	\item Most się tworzy, ale wije się na boki.
	\item Most się tworzy, ale skręca się wzdłuż, jak wstążka.
	\item Most się tworzy, ale w połowie się urywa.
	\item Nic się nie dzieje.
	\item Tworzą się jedynie fragmenty.
	\item Tworzą się tylko granice, jak szyny. 
	\item Most jest dziwnie lepki.
	\item Most jest tak śliski, że nie da się go używać.
	\item Most tworzy się w dziwnym kierunku, wykonaj test szczęścia \aba{}, czy kogoś przetnie na pół.
\end{enumerate}

\subsection{Szyfrator \abt{}}

\begin{tabular}{rl}
Wkłady & Szyfrator \\
Magazynek & 1 \\
Załadowanie na turę & 0,5 \\
\end{tabular}

Najwspanialsze urządzenie inżynierii potworów.
Podręczy pistolet, który zamraża wroga w czasie.
Można go odmrozić tym samym wkładem.
Działa to na zasadzie szyfrowania czasoprzestrzeni, klucz do odszyfrowania składowany jest w naboju.

Zmiana naboju trwa dwie rundy, najpierw trzeba wyciągnąć stary i włożyć nowy.
Wyrzucenie starego naboju wiąże się ze śmiercią zamrożonego (zostaje taki na zawsze) i dodaje 10 \abgrz{}.
Strzelenie ponownie tym samym nabojem nadpisuje klucz i dodaje 10 \abgrz{}.
Jeśli nie trafiło się w poprzedniej rundzie, można strzelać ponownie bez przeładowania.

Przy niezdaniu testu cechy nie ma żadnych efektów ubocznych!

\subsection{Baniecznica \abt{}}

\begin{tabular}{rl}
Wkłady & Wkłady do baniecznicy o różnych mocach \\
Magazynek & 10 \\
Załadowanie na turę & 2 \\
\end{tabular}

Baniecznica jest wstanie rozciągnąć na użytkowniku pole siłowe, przyjmujące dowolne obrażenia i zużywając wkłady.
Gdy poprawnie użyjesz baniecznicy, otrzymujesz pole siłowe aż do następnego obidu, blokujące wszystkie obrażenia na reakcję w wymiarze załadowania.
Przez powierzchnię nie można przejść i blokuje ona powietrze.
Jeśli pochłonie całe obrażenia, odbija je w losowym kierunku, zależnym od testu szczęścia \abh{}.
Sama aktywacja bańki nie zużywa energii.

Każdy wkład do baniecznicy ma różną moc, jest to ilość obrażeń, które pochłonie.
Jeśli wkład ma ujemną moc, zwiększa ilość obrażeń (mogą być nadal pochłonięte przez następne wkłady).
Wkłady są zużywane po kolei.
Po wykorzystaniu wkładu, kolejne obrażenia odejmowane są od następnego.
Naruszony wkład znika całkowicie, przykładowo wkład o mocy 5, przyjmujący 3 obrażenia, zniknie zupełnie, a nie zmieni się na wkład o mocy 2.

Po pochłonięciu wszystkich obrażeń, wykonaj test szczęścia \abh{} na to, gdzie odbije się skierowany w ciebie atak.
Wielki wynik odbija całość we wroga, lub przyjaciela. Mały odbija w innym kierunku.

Na przykład, jeśli mamy rząd wkładów o mocach: 4, 6, -1, 4, 10 i ktoś spróbuje zadać nam 11 punktów obrażeń, to pierwsze dwa wkłady pochłoną 10 obrażeń i zostanie tylko jedno.
Trzeci wkład o ujemnej wartości zwiększy ilość obrażeń do 2, które to zostaną pochłonięte przez czwarty, który nie zużyje się całkowicie.
Jednak jako, że czwarty został naruszony, to także znika. Zostaje tylko ostatni wkład o mocy 10.

\begin{enumerate}
	\item Stworzona bańka jest bardzo malutka, gnieciesz się niemiłosiernie i w następnym obidzie zabraknie ci powietrza.
	\item Bańka tworzy się tak wielka, że obejmuje także najbliższą istotę (może być wróg).
	\item Powstaje tylko \diiii{} połówka bańki: 1-przednia, 2-tylna, 3-górna, 4-dolna.
	\item Bańka jest dziurawa i przepuści połowę obrażeń.
	\item Wszystkie załadowane wkłady scalają się w jeden o sumie ich mocy.
	\item Bańka zamiast objąć ciebie, obejmuje najbliższą osobę (ale nadal masz sterowanie).
	\item Najmniejszy załadowany wkład odwraca znak.
	\item Największy załadowany wkład odwraca znak.
	\item Wszystkie załadowane wkłady odwracają znaki.
	\item Nic się nie dzieje, ale tracisz największy załadowany wkład.
\end{enumerate}

\section{Chipy aktywne}
Rurka do oddychania pod wodą wkręca mi się w głowę.

\subsection{Ostrza z ręki \abh{}, 3 \abnkp{}}
Trochę, jak Volverine, ale lepiej.
Gracz wysuwa trzy krótkie kolce z przedramienia.
\begin{enumerate}
	\item Tym razem wysuwa ci się tylko środkowy.
	\item Tym razem wysuwają ci się tylko boczne.
	\item Środkowy kolec wypada na podłogę.
	\item Środkowy kolec wystrzeliwuje, test szczęścia \aba{}, czy kogoś trafi.
	\item Kolce przestają się wysuwać w ogóle.
	\item Kolce przestają się chować w ogóle.
	\item Wszystkie trzy kolce wystrzeliwują we wszystkie strony, wykonaj testy szczęścia \aba{}.
	\item Kolce wysuwają się w drugą stronę, w rękę, -5 \abzyc{}.
	\item Zaczynają ciągle szybko wsuwać się i wysuwać.
	\item Wysuwają się od teraz tylko do połowy.
\end{enumerate}



% TODO Przykładowe chipy wraz z opisem uszkodzeń, efektów, cech

% TODO Mutacje nietypowe

% TODO Przedmioty

% TODO Urządzenia

% TODO Wiedza

% TODO Przykładowe skażenia uniwersalnością
% Struny w żyłach leczące rany
% Wytwarzanie osobnych części ciała kosztem życia
