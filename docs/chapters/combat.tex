\chapter{Walka}
Kolejność działań zależy od odwrotności \abp{}.
Im większa odległość od punktu osobowości do wierzchołka \abp{}, tym wolniej gracz reaguje.

W każdym ruchu gracz może:
\begin{itemize}
\item Wykonać jakąś akcję używającą cech. Można pomóc sobie mutacją, albo cechą.
\item Użyć broni wykonując jej test, a jeśli się nie uda, wylosować konsekwencje.
\item Użyć chipu wykonując test \abnkp{}, oraz jeśli się uda test cechy.
\item Wykonać test mocy, a jeśli się nie uda, wylosować konsekwencje spalenia.
\end{itemize}

Następnie musi zajrzeć do swojej listy pasywnej i wykonać wszystkie zawarte tam akcje, jak automatyczne ładowanie, leczenie ran itp.
Może to robić równolegle do drugiego gracza, bo nie wpływają one na inne osoby.


