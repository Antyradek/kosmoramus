\chapter{Walka}

Cała walka odbywa się w cyklach.
W jednym cyklu postać może wykonać kilka akcji, jedna akcja to jeden obid.
Kolejność tur jest ustalana losowo.

Przy uderzeniu przez wroga, możesz wykonać reakcję.

\section{Rozpoczęcie}
\begin{enumerate}
	\item Do puli obidów, każdy dodaje swoje własne, oraz \dvi{} nowych. Dodawane są obidy wrogów.
	\item Obidy się miesza i ustala losową permutację. Można na przykład wsadzić je do woreczka i po kolei ustawiać w linii.
	\item Każdy wykonuje test \abp{}, jeśli mu się uda, może dodać kolejny obid w ustalonym przez siebie miejscu.
\end{enumerate}
Istnieją różne modyfikacje dodające, lub odejmujące obidy przed i w trakcie gry.

\section{Akcja}
Gdy czas dojdzie do twojego obidu, możesz wykonać akcję.
\begin{itemize}
	\item Wykonać jakąś akcję używającą cech.
	\item Umieścić jeden wkład w broni.
	\item Opróżnić magazynek z broni.
	\item Użyć broni.
	\item Użyć chipu.
	\item Użyć mutacji.
	\item Użyć mocy, lub hipermocy.
	\item Przygotować się.
\end{itemize}
Przygotowanie się oznacza dodatkowe ułatwienia do ewentualnej reakcji, lub następnej akcji o 1. Bonusy z kilku przygotowań pod rząd się dodają.
\begin{description}
	\item[Celowanie] Dla każdej broni, lub strzelającego obiektu ułatwia jej test.
	\item[Skupienie] Ułatwia test mocy, ale nie towarzyszący test cechy.
	\item[Zmrużenie oczu w geście nienawiści] Ułatwia następny test \abh{}.
	\item[Przekrzywienie głowy na bok] Ułatwia następny test \abt{}.
	\item[Napięcie mięśni] Ułatwia następny test \abs{}.
	\item[Przypomnienie sobie wzoru skróconego mnożenia] Ułatwia następny test \abi{}.
	\item[Podskakiwanie w miejscu] Ułatwia następny test \aba{}.
	\item[Wysunięcie głowy do przodu] Ułatwia następny test \abp{}.
\end{description}


Następnie gracz musi zajrzeć do swojej listy pasywnej i wykonać wszystkie zawarte tam akcje, jak automatyczne ładowanie, leczenie ran itp.
Może to robić równolegle do drugiego gracza, bo nie wpływają one na inne osoby.

\section{Reakcja}
Jeśli ktoś trafi postać gracza, może on wykonać reakcję.
\begin{itemize}
	\item Unik, to test \aba{}, zdany uniknie obrażeń. Wcześniejsze przygotowanie ułatwia test o \diiii{}.
	\item Blok, testem \abs{}, zmniejsza obrażenia brutto o \dvi{}. Wcześniejsze przygotowanie zmniejsza o \dx{}.
\end{itemize}

\section{Wypadnięcie}
Jeśli stracisz wszystkie obidy w cyklu, nie będziesz mógł wykonywać akcji.

Za każdym razem, będziesz mógł wykonać test \abp{} i wrócić do rozgrywki w aktualnym czasie, dodając swój obid.


