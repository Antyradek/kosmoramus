\chapter{Tabele}
Tabele z których losuje się wydarzenia.

\section{Spalenie}
Wykonywane przy zwyczajnym użyciu mocy jeśli test mocy jest niezdany, ale test cechy jest zdany.
Rzuć \dc{} i sprawdź, co cię czeka.

\begin{enumerate}
	\item Pierwsze przedmioty w ekwipunku u ciebie i osoby stojącej najbliżej, zamieniają się miejscami.
	\item Rzucasz się, aby uściskać najbardziej kolczastą, lub nieprzyjemną w dotyku postać. Test \abh{} uchroni cię.
	\item Doświadczasz halucynacji przez \dvi{} tur.
	\item Nagle wymiotujesz.
	\item Wszyscy wokół wymiotują, oprócz ciebie.
	\item Łączysz się telepatycznie z najbliższą tobie osobą na \dvi{} tur. Każda następna akcja twoja i jego będzie wykonana przez obie wasze postacie.
	\item Najbliższy ciebie mechanizm staje w miejscu.
	\item Ładunki we wszystkich twoich załadowanych broniach znikają.
	\item Przy następnym użyciu mocy, wszystkie twoje cechy mocy będą o jeden większe.
	\item Twój następny test \abi{} będzie zawsze zdany.
	\item Twój następny test \abi{} będzie zawsze niezdany.
	\item Twój następny test \abs{} będzie zawsze zdany.
	\item Twój następny test \abs{} będzie zawsze niezdany.
	\item Twój następny test \abp{} będzie zawsze zdany.
	\item Twój następny test \abp{} będzie zawsze niezdany.
	\item Twój następny test \aba{} będzie zawsze zdany.
	\item Twój następny test \aba{} będzie zawsze niezdany.
	\item Twój następny test \abh{} będzie zawsze zdany.
	\item Twój następny test \abh{} będzie zawsze niezdany.
	\item Twój następny test \abt{} będzie zawsze zdany.
	\item Twój następny test \abt{} będzie zawsze niezdany.
	\item Otrzymujesz jedno szczęście.
	\item Wszyscy oprócz ciebie tracą jedno \abzyc{}, a ty zyskujesz sumę zabranych wartości.
	\item Osoba z najmniejszą ilością \abzyc{} w grupie leczy się za \dvi{}.
	\item Osoba z najmniejszą ilością \abzyc{} w grupie traci \dvi{} lub aż do uzyskania wartości 1.
	\item Udaje ci się pomimo to poprawnie użyć mocy.
	\item Dodaj sobie \dvi{} punktów \abzyc{}, a jeśli masz maksymalną ilość, to tyle odejmij.
	\item Odejmij sobie połowę aktualnej wartości \abzyc{}.
	\item Odejmij sobie \diiii{} punktów \abzyc{}.
	\item Dodaj sobie \diiii{} punktów \abzyc{}.
	\item Twój najbardziej bezużyteczny przedmiot się duplikuje.
	\item Tracisz węch na \dvi{} tur.
	\item Tracisz wzrok na \dvi{} tur.
	\item Tracisz władzę w nogach na \dvi{} tur.
	\item Otrzymujesz \diiii{} utrudnień do wszystkiego w następnej turze.
	\item Otrzymujesz \diiii{} ułatwień do wszystkiego w następnej turze.
	\item Musisz coś natychmiast zjeść, lub tracisz \diiii{} punktów \abzyc{}.
	\item Wszystkie włosy/łuski na twoim ciele odpadają.
	\item Tajemnicza siła gwałtownie obraca cię o 180\textdegree w prawo.
	\item Dostajesz niekontrolowany ślinotok na \dxx{} tur.
	\item Robisz się brzydki na dobę i otrzymujesz \diiii{} utrudnień \abt{}.
	\item Zaczyna być słychać muzykę, a ty niekontrolowanie tańczysz przez \dvi{} tur.
	\item Losowy obiekt pojawia się i spada ci na głowę.
	\item Ostatni przedmiot w ekwipunku najdalszej od ciebie osoby pojawia się nad twoją głową.
	\item Najcięższy przedmiot w ekwipunku najbliższej ci osoby spada ci na głowę.
	\item Teleportujesz i pojawiasz się nad głową najdalszej ci osoby.
	\item Najcięższa osoba z waszej grupy teleportuje się nad głowę najbliższego ci wroga.
	\item Fala wody zalewa podłogę wokół ciebie. Masz mokre buty.
	\item Na twoich nogach pojawiają się bryły lodu. \dx{} utrudnień \aba{}. \abs{}, aby je rozbić.
	\item Fala uderzeniowa odrzuca wszystkich wokół.
	\item Powstaje bańka o promieniu \diiii{} metrów w której czas staje dla wszystkich oprócz ciebie, jej promień zmniejsza się o metr na turę.
	\item Klonujesz się, a klon chce was zabić.
	\item Pomieszczenie, w którym się znajdujesz, wypełnia się próżnią.
	\item Wszystkie twoje bronie w ekwipunku wymieniają się z osobą stojącą najdalej.
	\item Twój najlżejszy przedmiot zostaje zassany do bocznej warstwy i znika.
	\item Wokół ciebie strzelają fajerwerki i sypie się brokat. Ktokolwiek je doceni i zda test \abt{} otrzyma \dvi{} ułatwień w następnej turze.
	\item Pierwsza z twoich broni strzela w losowym kierunku.
	\item Ostatnia z twoich broni strzela w losowym kierunku.
	\item Tracisz świadomość i zachowujesz się, jak kura na \diiii{} tur.
	\item Twoje ubranie zostaje zassane do bocznej warstwy i zostajesz goły.
	\item Zaczynasz pokładać się ze śmiechu, pamiętaj, że śmiech jest zaraźliwy, niech inni rzucają test \abh{} na to, czy uda im się wytrzymać.
	\item Tracisz całkowite tarcie o podłoże na \diiii{} tur.
	\item Odwracasz swoją płeć aż do następnego snu.
	\item Wszystkie zegary i wskaźniki w okolicy psują się i stają.
	\item Najbliższy kosz ze śmieciami wybucha, rozrzucając zawartość.
	\item Rozlega się wkoło ryk anielskich trąb końca świata, na szczęście to tylko test alarmu. Wszyscy testują swoje \abh{}, kto nie zda, ma \dvi{} utrudnień w następnej turze.
	\item Porastają cię korzenie, jesteś unieruchomiony tak długo, aż zdasz test \abs{}, albo ktoś ciebie uwolni swoim testem.
	\item Przypomina ci się zadanie matematyczne z podstawówki i zajmuje ci całą twoją uwagę, jesteś unieruchomiony, aż ją rozwiążesz testem \abi{}. Jeśli ktoś ci pomoże i także zda \abi{}, otrzymasz 1 ułatwienie.
	\item Przypominasz sobie o okrutności świata i płaczesz do póki nie zdasz testu \abh{}. Ktoś inny może cię pocieszyć swoim testem \abt{}, wtedy otrzymasz 1 ułatwienie, ale jeśli nie zda, otrzymasz 1 utrudnienie.
	\item Przypominasz sobie o swojej depresji. Jeśli przez następne \dxx{} tur nie zdasz jakiegoś testu \abt{}, popełnisz samobójstwo. Inni mogą cię unieruchomić.
	\item Stoisz i podziwiasz otoczenie do póki się nie napatrzysz i nie zdasz \abp{}. Inni mogą pomóc ci w patrzeniu swoimi testami \abp{}, wtedy dostaniesz 1 ułatwienie.
	\item Stwierdzasz, że to dobra pora na test twojego kroku tanecznego. Tańczysz, aż zatańczysz i zdasz \aba{}. Jeśli ktoś chce ci pomóc, może cię nauczyć testem \aba{}, wtedy otrzymasz 1 ułatwienie.
	\item Twoja grawitacja się odwraca na \diiii{} tur.
	\item Przez \diiii{} tur działa na ciebie siła równa grawitacji w losowym kierunku.
	\item Zaczynasz mówić wstecz na \diiii{} tur.
	\item Na czas \diiii{} tur zaczyna na was działać przyciągająca siła i tulicie się wszyscy do siebie.
	\item Najbliższe drzwi/właz/klapa otwierają się nagle na oścież.
	\item Najbardziej bezużyteczne urządzenie wybucha zadając ci \diiii{} punktów \abzyc{}.
	\item Najbliższa ciebie osoba zsikuje się.
	\item Przepiękny zapach roznosi się po pokoju. Dla każdego zdanie testu \abt{} ułatwia wszystkie testy w tym pomieszczeniu o 1.
	\item Teleportujesz się o \diiii{} metrów w losowym kierunku.
	\item Niekontrolowanie używasz \absm{} w losowym kierunku.
	\item Niekontrolowanie używasz \abrm{}.
	\item Niekontrolowanie używasz \abdm{}.
	\item Nagle zamieniasz się miejscami z najdalszą od ciebie postacią.
	\item Wykonaj następny ruch najbliższej ci postaci, zamiast ruchu twojej postaci, a gracz tamtej wykona następny ruch twojej postaci.
	\item Gracz najbliższej ci postaci musi wykonać \dxx{} przysiadów.
	\item Następnym razem wykonasz dwa ruchy, zamiast jednego.
	\item Najdalsza od ciebie osoba nagle przyciąga się do ciebie. Możesz ją złapać za pomocą \abs{}, lub uniknąć za pomocą \aba{}, ale wtedy uderzy się i straci życie.
	\item Twoja grawitacja znika całkowicie na \diiii{} tur. Za każdą turą wykonaj dodatkowy test \abp{} na to, czy nie zwymiotujesz.
	\item Zaczęło kręcić ci się w głowie. Otrzymujesz \diiii{} utrudnień \abp{}, zmniejszających się o 1, co turę.
	\item Możesz spróbować zdać test za kogoś swoim kryształem osobowości, ale nadal on wykona akcję.
	\item Pojawia się wokół ciebie kula o promieniu \diiii{}, którą można przejść tylko na zewnątrz.
	\item Otrzymujesz jedną z myśli Mistrza Gry, zdecyduj, jaką.
	\item Otrzymujesz jedną z myśli najbliższego wroga.
	\item Następny rzut kostką będziesz mógł przestawić, ale tylko o jedną ścianę.
	\item Postaw swoją szklankę na planszy, Mistrz Gry zinterpretuje ją fabularnie.
	\item Następny niezdany test osobowości będziesz mógł naprawić testem odwrotnej cechy.
	\item Wszystkie twoje niezaładowane ładunki przekazujesz najbliższej osobie, która nie posiada kompatybilnej broni.
	\item Zachwycasz się okrągłością wyrzuconej liczby.
	
\end{enumerate}

\section{Hiperspalenie}
Wykonywane przy porażce w trakcie wykonywania hipermocy. Podobnie, jak przy spaleniu, rzucasz \dc{}.

\begin{enumerate}
	\item Dodaj sobie \dxx{} punktów \xpmcn{}.
	\item Dodaj sobie \dvi{} kartridży \abkar{}.
	\item Odejmij sobie \diiii{} kartridży \abkar{}.
	\item Wszyscy oprócz ciebie otrzymują \dvi{} punktów \xpmcn{}.
	\item Przekazujesz wszystkie punkty \xpmcn{} najbliższej osobie.
	\item Najbliższa osoba przekazuje wszystkie punkty \xpmcn{} tobie.
	\item Podwajasz liczbę punktów \xpmcn{}.
	\item Twoja najsłabsza cecha mocy rozwija się do następnej.
	\item Twoja najmocniejsza cecha mocy zwija się do poprzedniej.
	\item Ustaw swój punkt mocy w tym samym miejscu, co u osoby najbliżej ciebie.
	\item Ustaw swój punkt osobowości w tym samym miejscu, co u osoby najbliżej ciebie.
	\item Osoba najbliżej ciebie przestawia swój punkt mocy w miejsce, w jakim ma swój punkt mocy osoba najdalsza od ciebie.
	\item Przeskocz punktem mocy w kierunku \absm{}.
	\item Przeskocz punktem mocy w kierunku \abdm{}.
	\item Przeskocz punktem mocy w kierunku \abrm{}.
	\item Przeskocz punktem mocy w kierunku środka trójkąta.
	\item Otrzymujesz korupcję kryształu osobowości.
	\item Otrzymujesz korupcję trójkąta mocy.
	\item Otrzymujesz nową szynę na trójkącie mocy.
	\item Zamiast tego hiperspalenia, dzieje się normalne spalenie.
	\item Naciągnij wierzchołek \abi{}.
	\item Naciągnij wierzchołek \abs{}.
	\item Naciągnij wierzchołek \abt{}.
	\item Naciągnij wierzchołek \abh{}.
	\item Naciągnij wierzchołek \abp{}.
	\item Naciągnij wierzchołek \aba{}.
	\item Obciągnij wierzchołek \abi{}.
	\item Obciągnij wierzchołek \abs{}.
	\item Obciągnij wierzchołek \abt{}.
	\item Obciągnij wierzchołek \abh{}.
	\item Obciągnij wierzchołek \abp{}.
	\item Przy następnym użyciu mocy, wszystkie twoje cechy mocy będą o jeden większe.
	\item Przeskocz punktem osobowości w kierunku najbliższego wierzchołka.
	\item Odskocz punktem osobowości z kierunku najbliższego wierzchołka.
	\item Naciągnij wierzchołek \absm{}.
	\item Naciągnij wierzchołek \abdm{}.
	\item Naciągnij wierzchołek \abrm{}.
	\item Obciągnij wierzchołek \absm{}.
	\item Obciągnij wierzchołek \abdm{}.
	\item Obciągnij wierzchołek \abrm{}.
	\item Tracisz dwa razy więcej \abkar{}, niż teraz.
	\item Wylosuj dwa hiperspalenia, zamiast jednego.
	\item Najbardziej popsuty chip się samoczynnie naprawia.
	\item Najmocniejszy chip się uszkadza.
	\item Najsłabszy chip się uszkadza.
	\item Twój pierwszy chip na liście się odczepia i ląduje w ekwipunku jako przedmiot.
	\item Przestaw swój punkt osobliwości symetrycznie, równolegle do osi \abh{} - \abt{}.
	\item Przestaw swój punkt osobliwości symetrycznie, równolegle do osi \abi{} - \abs{}.
	\item Przestaw swój punkt osobliwości symetrycznie, równolegle do osi \aba{} - \abp{}.
	\item Dodaj sobie jeden \abnkp{}.
	\item Wszyscy, oprócz ciebie odejmują sobie jeden \abnkp{}.
	\item Otrzymujesz jedną mutację, wylosuj ją sobie
	\item Przestaw punkt osobliwości o 1 w kierunku \abi{}.
	\item Przestaw punkt osobliwości o 1 w kierunku \abs{}.
	\item Przestaw punkt osobliwości o 1 w kierunku \abp{}.
	\item Przestaw punkt osobliwości o 1 w kierunku \aba{}.
	\item Przestaw punkt osobliwości o 1 w kierunku \abt{}.
	\item Przestaw punkt osobliwości o 1 w kierunku \abh{}.
	\item Twój najpotężniejszy chip się psuje.
	\item Losujesz korupcję kryształu osobowości, ale wylosowany obszar czyścisz.
	\item Losujesz korupcję trójkąta mocy, ale wylosowany obszar czyścisz.
	\item Zyskujesz całkowicie drugi kryształ osobowości z innym punktem, przed wykonaniem akcji musisz wybrać z którego chcesz zdawać test.
	\item Rzuć swoim kryształem w górę, ścianę, która wypadnie, całą zamaluj korupcją.
	\item Usuń obecną turę z cyklu.
	\item Przestaw swoją turę o \dvi{} do przodu.
	\item Przestaw swoją turę o \dvi{} do tyłu.
\end{enumerate}
\emph{...ciąg dalszy nastąpi}

\section{Grzech śmiertelny}
\begin{longtabu}{ l l }
Rzut \dxx{}	&	Opis	\\
\hline
1	&	Twoja najmocniejsza cecha mocy spada do zera \\
2	&	Naciągnij wszystkie wierzchołki kryształu osobowości \\
3	&	Naciągnij wszystkie kąty trójkąta mocy \\
4	&	Wylosuj \dvi{} korupcji kryształu osobowości \\
5	&	Wylosuj \dvi{} korupcji trójkąta mocy \\
6	&	Tracisz wszystkie obecne \xpmcn{} \\
7	&	Zyskujesz \dvi{} problemów psychicznych \\
8	&	Tracisz moc i trójkąt całkowicie, wylosuj sobie nową zaczynając od zera \\
9	&	Tym razem nic się nie stanie, ale nie licz na więcej łaski \\
10	&	Dostajesz \diiii{} nowe mutacje \\
11  &	Dostajesz \dxx{} punktów \abnkp{} \\
12	&	Wszystkie twoje chipy się psują \\
13	&	Dodaj 10\% utraty 1 \abzyc{} do listy pasywnej \\
14	&	Dodaj 5\% utraty 2 \abzyc{} do listy pasywnej \\
15	&	Dodaj 10\% zepsucia się ostatnio używanego chipa do listy pasywnej \\
16	&	Korupcja wielkości 3 w miejscu punktu osobowości \\
17	&	Korupcja wielkości 4 w miejscu punktu mocy \\
18	&	Tracisz cały swój ekwipunek \\
19	&	Zapominasz połowę swoich wiedz \\
20	&	Wylosuj \dvi{} uszkodzeń ciała \\
\end{longtabu}

% Problem psychiczny
% Problemy takie, jak gapienie się w ładnych miejscach, hazard, strach przed wodą, miłość do ognia.
% 20

% Uszkodzenie ciała
% 20

% Mutacja
% zaawansowany system generowania mutacji, albo wybór z listy

\section{Ramy}
Te ramy informują o tym, jak powinny wyglądać moce do wpisania do karty mocy, informacje o mutacji, karty przedmiotów itp.

\subsection{Moc}
Każda moc ma 3 pola cech mocy, każda cecha mocy ma opis aktualnej cechy i koszt rozwoju.
\begin{itemize} 
	\item Trójkąt mocy.
	\item Opis mocy opisujący ogólnie jak działa moc, jak jej obiekty są tworzone.
	\item Ilość posiadanych \xpmcn{}.
	\item Ilość posiadanych \abkar{}.
	\item Opis użycia \absm{}. Jakie akcje można wykonać tym testem itp.
	\item Użycie \abdm{}. Co ciekawego da się zrobić, jeśli zdałbyś ten test.
	\item Reakcja \abrm{}, czyli co czujesz używając tej cechy
	\item Lista cech mocy wraz z ich kolejnymi wartościami i liczbą \xpmcn{} między dwoma wartościami cechy potrzebną do rozwoju.
\end{itemize}
Opcjonalnie:
\begin{itemize}
	\item Aktualne odległości punktu mocy od smaków, aby przyspieszyć liczenie.
	\item Aktualna wartość sumy trójkąta.
	\item Wartość testu do zdania dla hipersumy trójkąta.
\end{itemize}


\subsection{Przedmiot}
Przedmiot może być czymkolwiek, ma znaczenie głównie fabularne.
\begin{itemize}
 \item Nazwa.
 \item Ilość.
\end{itemize}

\subsection{Urządzenie}
Urządzenie działa pasywnie dając jakieś ciekawe zdolności.
Może być to tarcza, peleryna-niewidka, albo transport.
\begin{itemize}
\item Nazwa urządzenia.
\item Działanie urządzenia, czyli co ci daje. Możesz dopisać do listy pasywności.
\item Uszkodzenia urządzenia, jak reaguje na świat, co się może kiedy zepsuć i jak naprawić. Można dopisać do listy pasywności.
\end{itemize}

\subsection{Wiedza}
Jakaś wąska dziedzina w której specjalizuje się postać.
\begin{itemize}
 \item Nazwa.
 \item Cecha, którą trzeba zdać do użycia wiedzy.
\end{itemize}

\subsection{Chip}
Chip daje dodatkowe zdolności.
Aktywne wymagają do uruchomienia zdania testu cechy.
\begin{itemize}
 \item Cecha do zdania, po zdaniu testu \abnkp{}, lub kreska jeśli pasywny.
 \item Koszt zamontowania, ile \abnkp{} dodaje.
 \item Nazwa.
 \item Opis działania.
 \item Ewentualne ułatwienia przy przygotowaniu.
 \item Tabela opisująca 10 zepsuć chipu.
\end{itemize}

\subsection{Mutacja}
Jakaś zmiana ciała, która daje dodatkowe zdolności.
\begin{itemize}
 \item Nazwa.
 \item Opis, co zmienia u gracza, jak działa.
 \item Cecha do testu, lub przekreślony jeśli pasywny.
 \item Cena \abnkp{}. 
 \item Ewentualne ułatwienia przy przygotowaniu.
\end{itemize}

\subsection{Broń}
Coś z czego się strzela i co trzeba uzupełniać odpowiednimi nabojami.
\begin{itemize}
 \item Nazwa.
 \item Opis.
 \item Test cechy do użycia.
 \item Wielkość magazynka.
 \item Typ naboi.
 \item Dodatkowe zasady dotyczące ładowania.
 \item Opis ułatwień przy przygotowaniu.
 \item Ilość załadowanych naboi na jedną turę.
 \item Lista 10 zdarzeń przy niezdaniu testu cechy.
\end{itemize}











