\chapter{Tabele}
Tabele z których losuje się wydarzenia.

\section{Spalenia kartridży}
\begin{longtabu}{ l l }
Rzut \dc	&	Opis	\\
\hline
1			&	Dodaj sobie \dxx{} \xpmcn{}	\\
2			&	Wszyscy oprócz ciebie tracą jedno \abzyc{} \\
3			&	Osoba z najmniejszym \abzyc{} leczy się za \dvi{} \\
4			&	Rzucasz się, aby uściskać najbardziej kolczastą osobę w gronie \\
5			&	Pierwszy przedmiot w ekwipunku zamienia się miejscami z pierwszym najbliższej osoby \\
6			&	Dodaj sobie \diiii{} \abkar{}	\\
7			&	Wszyscy oprócz cienie dostają \dvi{} punktów \xpmcn{} \\
8			&	Twoja najsłabsza cecha mocy rozwija się do następnej \\
9			&	Twoja najmocniejsza cecha mocy zwija się do poprzedniej \\
10			&	Następnym jednym użyciem mocy, twoja najsłabsza cecha będzie maksymalna \\
11			&	Dodaj sobie \diiii{} \abzyc{}	\\
12			&	W twoim tyłku materializuje się korek dodupny \\
13			&	Wszystko oprócz graczy w kuli o promieniu \dvi{} znika \\
14			&	Doświadczasz halucynacji przez \dvi{} godzin \\
15			&	Najdalsza od ciebie osoba wymiotuje \\
16			&	Odejmij sobie \diiii{} \xpmcn{} \\
17			&	Przeskocz punktem mocy w kierunku \absm{} \\
18			&	Przeskocz punktem mocy w kierunku \abdm{} \\
19			&	Przeskocz punktem mocy w kierunku \abrm{} \\
20			&	Wylosuj korupcję kryształu osobowości \\
21			&	Nie dzieje się absolutnie nic \\
22			&	Wylosuj dodatkowo jeszcze dwa spalenia \\
23			&	Twój najsłabszy przedmiot się duplikuje \\
24			&	Odejmij sobie \diiii{} \abkar{} \\
25			&	Utrudnij i naciągnij wierzchołek \abi{} \\
26			&	Utrudnij i naciągnij wierzchołek \abh{} \\
27			&	Utrudnij i naciągnij wierzchołek \aba{} \\
28			&	Utrudnij i naciągnij wierzchołek \abt{} \\
29			&	Utrudnij i naciągnij wierzchołek \abp{} \\
30			&	Utrudnij i naciągnij wierzchołek \abs{} \\
31			&	Odejmij sobie \diiii{} \abzyc{} \\
32			&	Przeskocz punktem osobowości odwrotnym kierunku do najbliższego wierzchołka \\
33			&	Wylosuj korupcję trójkąta mocy 	\\
34			&	Przeskocz punktem osobowości w kierunku najbliższego wierzchołka \\
35			&	Ułatw grę i obciągnij \abp{} \\
36			&	Ułatw grę i obciągnij \abt{} \\
37			&	Ułatw grę i obciągnij \abh{} \\
38			&	Ułatw grę i obciągnij \abi{} \\
39			&	Ułatw grę i obciągnij \abs{} \\
40			&	Ułatw grę i obciągnij \aba{} \\
41			&	Wylosuj nową szynę na trójkącie mocy \\
42			&	Naciągnij wierzchołek \absm{} utrudniając grę \\
43			&	Naciągnij wierzchołek \abdm{} utrudniając grę \\
44			&	Naciągnij wierzchołek \abrm{} utrudniając grę \\
45			&	Tracisz węch na \dc{} godzin \\
46			&	Wylosuj kolejne spalenie odejmując jeden \abkar{} \\
47			&	Obciągnij wierzchołek \abdm{} ułatwiając grę \\
48			&	Obciągnij wierzchołek \absm{}	ułatwiając grę \\
49			&	Obciągnij wierzchołek \absm{}	ułatwiając grę \\
50			&	Robisz się strasznie głodny, musisz zaraz coś zjeść, lub odjąć \abzyc{} \\
51			&	Magiczna naprawa najmniej popsutego chipu \\
52			&	Wszystkie włosy na twoim ciele odpadają \\
53			&	Zostajesz gwałtownie obrócony w prawo o 180\textdegree \\
54			&	Tracisz władzę w nogach na \dxx{} minut \\
55			&	Niekontrolowany ślinotok przez \dvi{} godzin \\
56			&	Zepsucie się najsłabszego chipu	\\
57			&	Słychać muzykę, a ty zaczynasz tańczyć wbrew swojej woli przez \dvi{} minut \\
58			&	Losowy obiekt pojawia się i spada ci na głowę \\
59			&	Fala wody pojawia się od ciebie i zalewa wszystko wokół \\
60			&	Twoje stopy pokrywają się bryłami lodu \\
61			&	Fala uderzeniowa odrzuca wszystkich wokół \\
62			&	Powstaje bańka o promieniu \diiii{} m w której czas staje oprócz dla ciebie \\
63			&	Powstaje bańka o promieniu \dxx{} m w której czas staje oprócz dla ciebie \\
64			&	Najbliższa osoba klonuje się, a klon chce was zabić \\
65			&	Przestaw punkt osobowości symetrycznie wzdłuż osi \abh{} --- \abt{} \\
66			&	Pomieszczenie wypełnia się próżnią \\
67			&	Dodaj sobie jeden \abnkp{} \\ 
68			&	Twój cały ekwipunek wymienia się z najdalej stojącą osobą \\
69			&	Twój najlżejszy przedmiot zostaje zassany do bocznej warstwy i znika \\
70			&	Wokół ciebie strzelają fajerwerki i sypie się brokat \\
71			&	Pierwsza broń strzela w losowym kierunku \\
72			&	Ślepniesz na \dvi{} minut \\
73			&	Tracisz świadomość i zachowujesz się jak kura na \dxx{} minut \\
74			&	Zyskujesz nową mutację, wylosuj ją sobie \\
75			&	Twoje ubranie zostaje zassane do bocznej warstwy, zostajesz goły \\
76			&	Ostatnia broń strzela w losowym kierunku \\
77			&	Zaczynasz pokładać się ze śmiechu na jedną turę \\
78			&	Tracisz tarcie o podłoże na \diiii{} minut \\
79			&	Odwracasz swoją płeć do następnego wschodu gwiazdy \\
80			&	Przestaw punkt osobowości symetrycznie wzdłuż osi \aba{} --- \abp{} \\
81			&	Słyszysz utrudniające szepty, przestaw punkt osobowości o 1 do \aba{} \\
82			&	Zdajesz sobie z czegoś sprawę, przestaw punkt osobowości o 1 do \abi{} \\
83			&	Wszystkie zgary i wskaźniki w okolicy się psują i stają \\
84			&	Najbliższy kosz ze śmieciami wybucha rozrzucając zawartość \\
85			&	Przed tobą pojawia się losowy obiekt \\
86			&	Rozlega się ryk trąb jerychońskich \\
87			&	Porasta cię trochę bluszczu i kwiaty \\
88			&	Twoja grawitacja odwraca się na \diiii{} minut \\
89			&	Na \diiii{} minut działa na ciebie siła w losowym kierunku \\
90			&	Zaczynasz mówić wstecz na \diiii{} tur \\
91			&	Na \dxx{} działa na was przyciągająca siła i tulicie się do siebie \\
92			&	Najbliższe drzwi/właz/klapa otwierają się na oścież \\
93			&	Najmniej skomplikowane technologicznie urządzenie wybucha i odejmuje 1 \abzyc{} \\
94			&	Najbliższa osoba zsikuje się \\
95			&	Chmura przepięknego zapachu roznosi się na boki \\
96			&	Przestaw punkt osobowości symetrycznie wzdłuż osi \abs{} --- \abi{} \\
97			&	Najcięższy przedmiot z ekwipunku przenosi się do najdalszej osoby \\
98			&	Teleportujesz się kilka metrów w losowym kierunku \\
99			&	Zepsucie się najpotężniejszego chipu \\
100			&	Niekontrolowanie używasz Siły Mocy w losowym kierunku \\
\end{longtabu}

\section{Grzech śmiertelny}
\begin{longtabu}{ l l }
Rzut \dxx{}	&	Opis	\\
\hline
1	&	Twoja najmocniejsza cecha mocy spada do zera \\
2	&	Naciągnij wszystkie wierzchołki kryształu osobowości \\
3	&	Naciągnij wszystkie kąty trójkąta mocy \\
4	&	Wylosuj \dvi{} korupcji kryształu osobowości \\
5	&	Wylosuj \dvi{} korupcji trójkąta mocy \\
6	&	Tracisz wszystkie obecne \xpmcn{} \\
7	&	Zyskujesz \dvi{} problemów psychicznych \\
8	&	Tracisz moc i trójkąt całkowicie, wylosuj sobie nową zaczynając od zera \\
9	&	Tym razem nic się nie stanie, ale nie licz na więcej łaski \\
10	&	Dostajesz \diiii{} nowe mutacje \\
11  &	Dostajesz \dxx{} punktów \abnkp{} \\
12	&	Wszystkie twoje chipy się psują \\
13	&	Dodaj 10\% utraty 1 \abzyc{} do listy pasywnej \\
14	&	Dodaj 5\% utraty 2 \abzyc{} do listy pasywnej \\
15	&	Dodaj 10\% zepsucia się ostatnio używanego chipa do listy pasywnej \\
16	&	Korupcja wielkości 3 w miejscu punktu osobowości \\
17	&	Korupcja wielkości 4 w miejscu punktu mocy \\
18	&	Tracisz cały swój ekwipunek \\
19	&	Zapominasz połowę swoich wiedz \\
20	&	Wylosuj \dvi{} uszkodzeń ciała \\
\end{longtabu}

% Problem psychiczny
% 20

% Uszkodzenie ciała
% 20

\section{Ramy}
Te ramy informują o tym, jak powinny wyglądać moce do wpisania do karty mocy, informacje o mutacji, karty przedmiotów itp.

\subsection{Moc}
Każda moc ma 5 pól cech mocy, każda cecha mocy ma opis aktualnej cechy i koszt rozwoju.
\begin{itemize} 
\item Trójkąt mocy.
\item Opis mocy opisujący ogólnie jak działa moc, jak jej obiekty są tworzone.
\item Opis użycia \absm{}. Jakie akcje można wykonać tym testem itp.
\item Użycie \abdm{}. Co ciekawego da się zrobić, jeśli zdałbyś ten test.
\item Reakcja \abrm{}, czyli co czujesz używając tej cechy
\item Lista cech mocy wraz z ich kolejnymi wartościami i liczbą \xpmcn{} między dwoma wartościami cechy potrzebną do rozwoju.
\end{itemize}

\subsection{Przedmiot}
Przedmiot może być czymkolwiek, ma znaczenie głównie fabularne.
\begin{itemize}
 \item Nazwa.
 \item Ilość.
\end{itemize}

\subsection{Urządzenie}
Urządzenie działa pasywnie dając jakieś ciekawe zdolności.
Może być to tarcza, peleryna-niewidka, albo transport.
\begin{itemize}
\item Nazwa urządzenia.
\item Działanie urządzenia, czyli co ci daje. Możesz dopisać do listy pasywności.
\item Uszkodzenia urządzenia, jak reaguje na świat, co się może kiedy zepsuć i jak naprawić. Można dopisać do listy pasywności.
\end{itemize}

\subsection{Wiedza}
Jakaś wąska dziedzina w której specjalizuje się postać.
\begin{itemize}
 \item Nazwa.
 \item Cecha, którą trzeba zdać do użycia wiedzy.
\end{itemize}

\subsection{Chip}
Chip daje dodatkowe zdolności.
Aktywne wymagają do uruchomienia zdania testu cechy.
\begin{itemize}
 \item Cecha do zdania po zdaniu testu \abnkp{}, lub kreska jeśli pasywny.
 \item Koszt zamontowania, ile \abnkp{} dodaje.
 \item Nazwa.
 \item Opis działania.
 \item Tabela opisująca 10 zepsuć chipu.
\end{itemize}

\subsection{Mutacja}
Jakaś zmiana ciała, która daje dodatkowe zdolności.
\begin{itemize}
 \item Nazwa.
 \item Opis, co zmienia u gracza, jak działa.
 \item Cecha do testu, lub przekreślony jeśli pasywny.
 \item Cena \abnkp{}. 
\end{itemize}

\subsection{Broń}
Coś z czego się strzela i co trzeba uzupełniać odpowiednimi nabojami.
\begin{itemize}
 \item Nazwa.
 \item Opis.
 \item Test cechy do użycia.
 \item Wielość magazynka.
 \item Typ naboi.
 \item Dodatkowe zasady dotyczące ładowania.
 \item Ilość załadowanych naboi na jedną turę.
 \item Lista 10 zdarzeń przy niezdaniu testu cechy.
\end{itemize}











