\chapter{Opis gracza}
\section{Cechy gracza}
\kosmoramus to gra fabularna. Są w niej gracze a nad wszystkim czuwa Mistrz Gry.

Gracz posiada 6 cech podstawowych za pomocą których wykonuje testy na umiejętności i używanie broni.
Cechy są modyfikowane przez chipy i różne inne dodatki.
\begin{description}
 \item[Siła \abs] Siła decyduje o tym jak silny i potężny jest gracz.
 \item[Inteligencja \abi] Jak bardzo udaje mu się rozwiązywać problemy i co potrafi wymyślić.
 \item[Chamstwo \abh] Odwaga, bezwzględność i odporność. Chamski gracz ma wysoką siłę woli.
 \item[Towarzyskość \abt] Altruizm, rozmowy z innymi, dar przekonywania. Towarzyskie osoby mogą dobrze zarządzać innymi i organizować.
 \item[Atletyzm \aba] Skoczność, gibkość i unik. Atleci potrafią robić wspaniałe rzeczy, ale przez bardzo krótki okres czasu.
 \item[Percepcja \abp] Gracze z wysoką percepcją widzą więcej i mogą się skupić. Potrzebna ona jest do dokładnego, acz wolnego sterowania.
\end{description}

Każda cecha odpowiada określonemu typowi człowieka.
Zatem są one przeciwstawne parami, rozwijanie jednej cechy jest kosztem naprzeciwległej.
\begin{itemize}
 \item \abs jest naprzeciwko \abi.
 \item \abh jest naprzeciwko \abt.
 \item \aba jest naprzeciwko \abp.
\end{itemize}

Najprościej jest zaprezentować zdolności gracza na wierzchołkach ośmiościanu foremnego zwanego kryształem osobowości:
\begin{itemize}
 \item Cechy przeciwstawne są na przeciwstawnych wierzchołkach, na każdej osi układu współrzędnych jedna para.
 \item Bok ośmiościanu powinien wynosić 10.
 \item Na każdym boku jest siatka z trójkątów po której gracz liczy test cechy.
\end{itemize}

Dodatkowo każdy z bohaterów posiada:
\begin{description}
 \item [Wskaźnik niekompatybilności \abnkp] Jak bardzo ciało gracza jest niespójne ze sobą. Chipy i mutacje obniżają \abnkp w różny sposób. Rasowo czyste istoty mają zerową \abnkp.
 \item [Życie \abzyc] Standardowe punkty obrażeń, ile jest w stanie gracz znieść.
 \item [Wskaźnik grzechu \abgrz] Za złe uczynki, które nie są konieczne do wykonania misji zwiększa się tą ilość. Gdy osiągnie 100 następuje grzech śmiertelny.
\end{description}

\section{Testy}
Rzucając jedną \dxx liczymy odległość punktu naszego gracza od wierzchołka po trójkątnej kratce, jeśli jest mniejsza niż wyrzucona wartość, to test jest zdany.

Mistrz gry może dodać utrudnienia i ułatwienia dla zdawalności testu pod postacią dodatkowych, lub ujemnych punktów do wyniku na kostce.
Dodanie punktów do wyniku na kostce zwiększa trudność wyrzucenia mniejszej wartości od odległość na krysztale osobowości, a odjęcie ułatwia.

\section{Życie}
Ilość punktów życia gracza zależy od jego siły i jest równa $20 - \abs$.
Tyle uszkodzeń może on stracić w grze i nie wpłynie to na rozgrywkę.
Ta wartość może być modyfikowana przez chipy, mutacje, zbroje itp.
Jednak spadek poniżej zera powoduje rany krytyczne.

Każda kolejna rana krytyczna powoduje w kolejności:
\begin{enumerate}
\item Problem psychiczny.
\item Korupcja kryształu osobowości.
\item Uszkodzenia ciała.
\end{enumerate}

\section{Korupcja}
Korupcja kryształu to losowe zaznaczenie zniszczonego obszaru.
Liczenie po tej drodze umiejętności liczy się podwójnie.
Może się zdarzyć, że korupcja wpasuje się na wierzchołku, wtedy niestety każdy test będzie obarczony dodatkowym utrudnieniem.

Aby policzyć korupcję:
\begin{enumerate}
 \item Wybierz wierzchołek rzucając \dvi, ustaw go na górze.
 \item Idź w dół po krawędzi najbliższej do twojego punktu \dxx kratek.
 \item Idź w prawo \dxx[2] kratek.
 \item Zaciemnij wszystkie trójkąty w odległości 2 od wylosowanego punktu. Najczęściej jest to sześciokąt o boku 2, ale na wierzchołku będzie to kwadrat.
\end{enumerate}
Może się zdarzyć, że punkt gracza znajdzie się w zaciemnionym obszarze. 
Na szczęście cechy można modyfikować i w przyszłości uciec z utrudniającego obszaru.

\section{Chipy}
Chipy dzielą się na aktywne i pasywne.

Aktywne dają dodatkowe zdolności, jak umiejętność odbioru fal radiowych, lutownice w palcach, czy noktowizor w oczach.
Przed użyciem chipu należy przejść test \abnkp, czyli wyrzucić za pomocą \dxx więcej, niż wartość \abnkp. 
Niezdanie testu \abnkp powoduje widowiskowe uszkodzenie chipu z różnymi dziwacznymi skutkami.
Każdy chip związany jest także z pewną cechą, której to test trzeba zdać przed użyciem chipu..
Każdy zamontowany chip dodaje pewną ilość \abnkp dla gracza.

Chipy pasywne mogą obciągać wierzchołki kryształu osobowości, rysujemy wtedy kwadrat wokół odpowiedniego wierzchołka i liczymy drogę tylko do granicy.
W ten sposób ułatwiamy sobie zdawanie testów danej cechy.
Chipy pasywne także nie uszkadzają się i nie trzeba dla nich zdawać testów \abnkp, jednakże cena za ich zamontowanie często jest bardzo wysoka i mocno zwiększa ilość \abnkp.

Różne chipy dają różnie przydatne zdolności, niektóre mogą montować się automatycznie włażąc w ciało.
Inne mogą potrzebować operacji chirurgicznej, aby odpowiednio móc korzystać z jego dobrodziejstw.
Operacja może pójść w różny sposób, nieudana w pełni może wpłynąć tymczasowo na \abzyc, zmniejszyć moc chipu, albo zwiększyć nadmiernie \abnkp.

\section{Uszkodzenia chipu}
Przy nie zdaniu \abnkp chip może się uszkodzić na jeden ze sposobów, charakterystyczny dla każdego chipu.
Rzucamy wtedy \dxx i sprawdzamy swój wynik w tabeli pod informacjami o chipie.

Zazwyczaj jest to zaprzestanie działania, ale może on także zacząć wyprawiać dziwne rzeczy.
Możliwe, że zwiększy, lub zmniejszy się jego czułość.
Może dodać się szum, lub błędy losowane przez Mistrza Gry.
Czasem inne chipy mogą wpływać na używanie zepsutego.

Uszkodzenia mogą być tymczasowe na jedną, lub więcej tur --- w przyszłości na pewno będą istnieć samonaprawiające się układy cyfrowe, lub permanentne --- naprawiane na różne inne sposoby.

\section{Moc}
Każdy gracz posiada jedną supermoc.
Jest to światłograf podpisany z Bogiem, który wymaga od gracza działania w kierunku większego dobra i aktywnego zwalczania zła przy pomocy nabytej umiejętności.

%cechy mocy
%trójkąt mocy
%kartridże
%rozwinięcie mocy
%korupcja mocy
\section{Mutacje}

\section{Grzech}
%grzech śmiertelny
%losowanie efektu jak korupcja (mocy), spadek osobowości, dodatkowy problem fizyczny, blizny

\section{Broń}
%wkłady 
\section{Przedmioty}
%automatyczne ładowniki itp.
\section{Wiedza}
%zdobyte umiejęctności zwiększające zdanie testu
\section{Problem Psychiczny}
%ma problemy prychiczne w stresie, może je też dostawać
\section{Uszkodzenia ciała}
%blizny itp zależne od okoliczności, rzut kostką na wielkość uszkodzeń
%liczą się do wyglądu, ale mogą zwiększać niekompatybilność




